% This must be in the first 5 lines to tell arXiv to use pdfLaTeX, which is strongly recommended.
\pdfoutput=1
% In particular, the hyperref package requires pdfLaTeX in order to break URLs across lines.

\documentclass[11pt]{article}

% Remove the "review" option to generate the final version.
% \usepackage[review]{acl}
\usepackage{authblk}
\usepackage{acl}
% Standard package includes
\usepackage{times} 
\usepackage{latexsym}
\usepackage{multirow}
\usepackage{array}


% For proper rendering and hyphenation of words containing Latin characters (including in bib files)
\usepackage[T1]{fontenc}
% For Vietnamese characters
% \usepackage[T5]{fontenc}
% See https://www.latex-project.org/help/documentation/encguide.pdf for other character sets

% This assumes your files are encoded as UTF8
\usepackage[utf8]{inputenc}

\usepackage{enumitem}
\setlist{nolistsep,leftmargin=*}

% This is not strictly necessary, and may be commented out.
% However, it will improve the layout of the manuscript,
% and will typically save some space.
\usepackage{microtype}

% This is also not strictly necessary, and may be commented out.
% However, it will improve the aesthetics of text in
% the typewriter font.
\usepackage{inconsolata}
\usepackage{graphicx} 
\usepackage{float} 
\usepackage{subfigure} 


% If the title and author information does not fit in the area allocated, uncomment the following
%
%\setlength\titlebox{<dim>}
%
% and set <dim> to something 5cm or larger.

\title{573 Project}

% Author information can be set in various styles:
% For several authors from the same institution:
\author{{\bf Yu Ying Chiu},   {\bf Jiayi Yuan},   {\bf Yian Wang},   {\bf Chenxi Li} \\
% % \author{Author 1 \\ {\bf Author 2} \\ ... \\ {\bf Author n}
        Department of Linguistics \\ University of Washington \\ Seattle, WA, USA \\
        \texttt{\{kellycyy, jiayiy9, wangyian, cl91\}@uw.edu}} 

% \author[*]{Yu Ying Chiu}
% \author[*]{Jiayi Yuan}
% \author[*]{Yian Wang}
% \author[*]{Chenxi Li}
% \affil[*]{Department of Linguistics, University of Washington}
% \texttt{\{kellycyy, jiayiy9, wangyian, cl91\}@uw.edu}
\begin{document}
% it seems author info needs to be entered in .sty file, line 150
\maketitle
\begin{abstract}
This is the abstract part.
\end{abstract}

\section{Introduction}
% Briefly overview the questions you are approaching, summarize the main conclusions, and give an overview of the paper. \\
% % \citep: (author, year)
% % \citet: author(year)
% This is an example of citation. ... \citep{paul2010summarizing}.\\
This is the introduction part.

\section{Task description}
We select the shared task (FETA challenge) by \citet{albalak-etal-2022-feta}. It is a new benchmark for few-sample task transfer in open-domain dialogue. In this benchmark, we selected the FETA-Friends as our dataset from \citet{chen2016character}. It involves transcripts for all 10 seasons of the TV show (Friends). 

The FETA-friends benchmark contains 7 tasks. We selected Emotion Recognition (Emory NLP) by \citet{zahiri2017emotion} as our primary task and Personality Detection by \citet{jiang2020automatic} as our adaptation task.

For the primary task, the dataset contains utterances with one of the annotated emotions (eg. Neutral, Joyful, Powerful, Mad, Scared etc.). According to the dataset github \footnote{\url{https://github.com/emorynlp/emotion-detection}}, the dataset fields include the utterance id, speaker name, transcript, tokens and annotated emotion. It is an utterance-level classification task. The evaluation is to calculate the micro-F1 and W-F1 scores of the prediction (classification output). 

For the adaptation task, the dataset contains short conversations with annotated binary Big Five personality traits (Agreeableness, Conscientiousness, Extroversion, Openness, and Neuroticism). According to the dataset github \footnote{\url{https://github.com/emorynlp/personality-detection}}, each personality trait was annotated on a scale of -1, 0 , 1. The annotation of each trait in a short conversation will be summed up. The dataset contains scene id, character, AGR (Agreeableness), CON (Conscientiousness), EXT (Extroversion), OPN (Openness), NEU (Neuroticism), and text. It is a dialogue-level classification task. The evaluation is to calculate the accuracy of the prediction output.

For the key dimensions, the primary task has emotion as affect type, classification as recognition type, TV transcript as Genre, aspect-specific emotion as the target, text as modality, and English as language. For the adaptation task, it has personality type as affect type,  aspect-specific personality detection as target, and all the other dimensions are the same as the primary task selected.

The FETA benchmark \footnote{\url{https://github.com/alon-albalak/TLiDB/blob/master/FETA_README.md}} provides access to data (train, test, dev), training and evaluation tool as a reference. The original author of dataset \footnote{\url{https://github.com/emorynlp}} also provides access to the same dataset.

\section{System Overview}\label{sys-overview}
This provides an overview for the systems.

\section{Approach}\label{approach}
% Our approach is comprised of four main steps. 
% Pre-processing, content selection, IO, and content realization.
This is the approach part.


\section{Results}\label{results}
This presents the results of the model.

\section{Discussion}\label{discussion}
Discuss the results.

\section{Ethical Considerations}\label{ethical considerations}
This part discusses ethical considerations.

\section{Conclusions}\label{conclusion}
This part presents the conclusion.



% \notice{X, Y, ...} displays selected references from custom.bib
% X = string before author in custom.bib

% \nocite{albalak-etal-2022-feta}

% Entries for the entire Anthology, followed by custom entries
\bibliography{anthology,custom}
\bibliographystyle{acl_natbib}

\appendix
\section*{Appendix A: Work Split}
\label{sec:appendix}
work split in detail

\appendix
% section* to remove number/letter
\section*{Appendix B: Packages Used in the Systems}
\label{sec:appendix}
% \url{address} to add url
Link to the code repository on github: \\\url{https://github.com/kellycyy/LING573-project}\\\\
Off-the-shell tools used in code:

\begin{itemize}
\item \texttt{sample package} how the package is used for what
\end{itemize}

\end{document}

